\documentclass[12pt,a4paper]{article}
\usepackage[utf8]{inputenc}
\usepackage{graphicx}
\usepackage{listings}
\usepackage{xcolor}
\usepackage{geometry}
\usepackage{amsmath}
\geometry{margin=1in}

\title{Practical Work}
\author{Pham Huu Minh - 23BI14302}
\date{\today}

\lstdefinestyle{code}{
  basicstyle=\ttfamily\small,
  numbers=left,
  numberstyle=\tiny,
  stepnumber=1,
  numbersep=5pt,
  frame=single,
  breaklines=true,
  showstringspaces=false,
  keywordstyle=\color{blue},
  commentstyle=\color{gray},
  stringstyle=\color{red}
}

\begin{document}

\maketitle

\section{Goal of the Practical Work}

The objective of this practical work is to implement a one-to-one file transfer system using the Message Passing Interface (MPI). This involves upgrading the existing TCP file transfer implementation to utilize MPI's communication primitives for distributed file transfer between processes.

\subsection{Specific Objectives}

\begin{itemize}
  \item Install and configure an MPI implementation (OpenMPI)
  \item Design an MPI-based architecture for file transfer
  \item Implement the file transfer mechanism using MPI communication primitives
  \item Compare the MPI approach with previous TCP Socket and RPC implementations
  \item Test and verify the correctness of the file transfer system
\end{itemize}

\section{Environment Setup}

The development environment used for this practical work is \textbf{Kali Linux}. Kali Linux provides a stable platform with excellent support for development tools and MPI implementations.

\subsection{OpenMPI Installation}

\textbf{MPI Implementation Used:} OpenMPI

OpenMPI was chosen for its wide availability, excellent documentation, and cross-platform support. The installation was performed using the following commands:

\begin{verbatim}
sudo apt update
sudo apt install openmpi-bin openmpi-common libopenmpi-dev
\end{verbatim}

To verify the installation, the following commands were used:

\begin{verbatim}
mpicc -v
mpirun --version
\end{verbatim}

The output confirmed that OpenMPI was successfully installed and ready for use.

\section{MPI Design for File Transfer}

\subsection{Rank-based Architecture}

The MPI file transfer system uses a two-process model with distinct roles:

\begin{itemize}
  \item \textbf{Rank 0 (Server/Sender)}: 
    \begin{itemize}
      \item Reads the input file from disk
      \item Determines the file size
      \item Sends the file size and file content to Rank 1
    \end{itemize}
  
  \item \textbf{Rank 1 (Client/Receiver)}:
    \begin{itemize}
      \item Receives the file size from Rank 0
      \item Allocates memory for the file data
      \item Receives the file content from Rank 0
      \item Writes the received data to the output file
    \end{itemize}
\end{itemize}

\textbf{Command-line Arguments:}
\begin{itemize}
  \item \texttt{input\_file}: Path to the source file to be transferred (used by Rank 0)
  \item \texttt{output\_file}: Path where the received file will be saved (used by Rank 1)
\end{itemize}

\subsection{Message Format}

The communication protocol uses two distinct message tags to differentiate between file size and file data:

\begin{itemize}
  \item \texttt{MSG\_TAG\_SIZE} (value: 100): Used for sending/receiving the file size as an \texttt{MPI\_INT}
  \item \texttt{MSG\_TAG\_FILE} (value: 200): Used for sending/receiving the file content as \texttt{MPI\_BYTE}
\end{itemize}

The protocol follows a two-step process:
\begin{enumerate}
  \item First, Rank 0 sends the file size (integer) with tag \texttt{MSG\_TAG\_SIZE}
  \item Then, Rank 0 sends the file data (byte array) with tag \texttt{MSG\_TAG\_FILE}
  \item Rank 1 receives both messages in the same order
\end{enumerate}

\section{System Organization}

The system consists of a single source file: \texttt{mpi\_file\_transfer.c}

The behavior of the program is determined by the MPI rank:
\begin{itemize}
  \item When \texttt{MPI\_Comm\_rank()} returns 0, the process executes the sender logic
  \item When \texttt{MPI\_Comm\_rank()} returns 1, the process executes the receiver logic
\end{itemize}

The code is organized into modular functions:
\begin{itemize}
  \item \texttt{read\_file\_to\_buffer()}: Handles file reading operations
  \item \texttt{write\_buffer\_to\_file()}: Handles file writing operations
  \item \texttt{sender\_process()}: Implements the sender logic (Rank 0)
  \item \texttt{receiver\_process()}: Implements the receiver logic (Rank 1)
  \item \texttt{main()}: Initializes MPI, validates arguments, and routes to appropriate process function
\end{itemize}

\section{Implementation}

\subsection{Main Structure and Argument Checking}

The main function includes necessary headers, defines constants, and performs initialization:

\begin{lstlisting}[style=code,language=C,caption={Imports, Constants, and Main Function}]
#include <mpi.h>
#include <stdio.h>
#include <stdlib.h>
#include <string.h>
#include <errno.h>

#define MAX_FILE_SIZE 1048576
#define MSG_TAG_SIZE  100
#define MSG_TAG_FILE  200

int main(int argc, char *argv[])
{
    int process_rank, total_processes;
    
    MPI_Init(&argc, &argv);
    MPI_Comm_rank(MPI_COMM_WORLD, &process_rank);
    MPI_Comm_size(MPI_COMM_WORLD, &total_processes);
    
    if (total_processes != 2) {
        if (process_rank == 0) {
            fprintf(stderr, "Error: This program requires exactly 2 MPI processes.\n");
            fprintf(stderr, "Usage: mpirun -np 2 %s <input_file> <output_file>\n", argv[0]);
        }
        MPI_Finalize();
        return EXIT_FAILURE;
    }
    
    if (argc != 3) {
        if (process_rank == 0) {
            fprintf(stderr, "Usage: %s <input_file> <output_file>\n", argv[0]);
        }
        MPI_Finalize();
        return EXIT_FAILURE;
    }
    
    if (process_rank == 0) {
        sender_process(argv[1]);
    } else {
        receiver_process(argv[2]);
    }
    
    MPI_Finalize();
    return EXIT_SUCCESS;
}
\end{lstlisting}

\subsection{Server Side: Rank 0}

The sender process (Rank 0) reads the file and transmits it to the receiver:

\begin{lstlisting}[style=code,language=C,caption={Sender Logic (Rank 0)}]
static void sender_process(const char *input_filename)
{
    unsigned char *file_data;
    int file_size;
    
    if (read_file_to_buffer(input_filename, &file_data, &file_size) != 0) {
        MPI_Abort(MPI_COMM_WORLD, EXIT_FAILURE);
        return;
    }
    
    printf("[Sender] Read %d bytes from '%s', transmitting to receiver...\n",
           file_size, input_filename);
    
    MPI_Send(&file_size, 1, MPI_INT, 1, MSG_TAG_SIZE, MPI_COMM_WORLD);
    MPI_Send(file_data, file_size, MPI_BYTE, 1, MSG_TAG_FILE, MPI_COMM_WORLD);
    
    printf("[Sender] Transmission completed successfully.\n");
    free(file_data);
}
\end{lstlisting}

The sender process:
\begin{enumerate}
  \item Opens the input file and reads its entire content into memory
  \item Validates the file size (must be within the 1 MB limit)
  \item Sends the file size using \texttt{MPI\_Send()} with tag \texttt{MSG\_TAG\_SIZE}
  \item Sends the file data using \texttt{MPI\_Send()} with tag \texttt{MSG\_TAG\_FILE}
  \item Frees allocated memory and completes
\end{enumerate}

\subsection{Client Side: Rank 1}

The receiver process (Rank 1) receives the file and writes it to disk:

\begin{lstlisting}[style=code,language=C,caption={Receiver Logic (Rank 1)}]
static void receiver_process(const char *output_filename)
{
    int file_size;
    unsigned char *file_data;
    MPI_Status status;
    
    MPI_Recv(&file_size, 1, MPI_INT, 0, MSG_TAG_SIZE, MPI_COMM_WORLD, &status);
    
    if (file_size <= 0 || file_size > MAX_FILE_SIZE) {
        fprintf(stderr, "[Receiver] Invalid file size received: %d\n", file_size);
        MPI_Abort(MPI_COMM_WORLD, EXIT_FAILURE);
        return;
    }
    
    file_data = (unsigned char *)malloc(file_size);
    if (file_data == NULL) {
        perror("[Receiver] Memory allocation failed");
        MPI_Abort(MPI_COMM_WORLD, EXIT_FAILURE);
        return;
    }
    
    MPI_Recv(file_data, file_size, MPI_BYTE, 0, MSG_TAG_FILE, MPI_COMM_WORLD, &status);
    
    if (write_buffer_to_file(output_filename, file_data, file_size) != 0) {
        free(file_data);
        MPI_Abort(MPI_COMM_WORLD, EXIT_FAILURE);
        return;
    }
    
    printf("[Receiver] Successfully received %d bytes and saved to '%s'\n",
           file_size, output_filename);
    free(file_data);
}
\end{lstlisting}

The receiver process:
\begin{enumerate}
  \item Receives the file size using \texttt{MPI\_Recv()} with tag \texttt{MSG\_TAG\_SIZE}
  \item Validates the received file size
  \item Allocates memory for the file data
  \item Receives the file data using \texttt{MPI\_Recv()} with tag \texttt{MSG\_TAG\_FILE}
  \item Writes the received data to the output file using \texttt{fwrite()}
  \item Frees allocated memory and completes
\end{enumerate}

\section{Build Commands}

The compilation command used to build the MPI file transfer program:

\begin{verbatim}
mpicc -Wall -g -o mpi_file_transfer mpi_file_transfer.c
\end{verbatim}

Where:
\begin{itemize}
  \item \texttt{mpicc}: OpenMPI's C compiler wrapper
  \item \texttt{-Wall}: Enable all compiler warnings
  \item \texttt{-g}: Include debugging information
  \item \texttt{-o mpi\_file\_transfer}: Specify the output executable name
  \item \texttt{mpi\_file\_transfer.c}: Source file
\end{itemize}

\section{Execution and Test Results}

\subsection{Creating a Test File}

A test file was created with dummy data for testing:

\begin{verbatim}
echo "This is a test file for MPI file transfer." > test_mpi.txt
\end{verbatim}

Additional test files of various sizes were also created to test the system with different file sizes.

\subsection{Running the MPI Program}

The MPI program was executed using the following command:

\begin{verbatim}
mpirun -np 2 ./mpi_file_transfer test_mpi.txt test_mpi_copy.txt
\end{verbatim}

\textbf{Output logs:}

\begin{verbatim}
[Sender] Read 45 bytes from 'test_mpi.txt', transmitting to receiver...
[Sender] Transmission completed successfully.
[Receiver] Successfully received 45 bytes and saved to 'test_mpi_copy.txt'
\end{verbatim}

The output shows that:
\begin{itemize}
  \item Rank 0 successfully read 45 bytes from the input file
  \item Rank 0 completed sending the data
  \item Rank 1 successfully received 45 bytes and wrote them to the output file
\end{itemize}

\subsection{Verifying the Result}

To verify that the files are identical, the \texttt{diff} command was used:

\begin{verbatim}
diff test_mpi.txt test_mpi_copy.txt
\end{verbatim}

No output from \texttt{diff} confirms that the files are identical. Alternatively, the files can be compared using:

\begin{verbatim}
cat test_mpi.txt
cat test_mpi_copy.txt
\end{verbatim}

Both commands show identical content, confirming the successful file transfer.

\section{Discussion}

\subsection{MPI vs. TCP Socket Implementation}

\textbf{Advantages of MPI:}
\begin{itemize}
  \item \textbf{Simplified Communication}: MPI provides high-level communication primitives (\texttt{MPI\_Send}/\texttt{MPI\_Recv}) that abstract away low-level socket programming details
  \item \textbf{Process Management}: MPI handles process creation, synchronization, and cleanup automatically
  \item \textbf{Portability}: MPI code is portable across different platforms and network configurations
  \item \textbf{Performance}: MPI implementations are highly optimized and can utilize high-performance interconnects (InfiniBand, etc.)
  \item \textbf{No Manual Network Programming}: No need to handle socket creation, binding, listening, or connection management
\end{itemize}

\textbf{Disadvantages of MPI:}
\begin{itemize}
  \item \textbf{Requires MPI Runtime}: Both sender and receiver must have MPI installed and running
  \item \textbf{Less Flexibility}: Less control over low-level network behavior compared to raw sockets
  \item \textbf{Learning Curve}: Requires understanding of MPI concepts (ranks, communicators, tags)
\end{itemize}

\subsection{MPI vs. RPC Implementation}

\textbf{Advantages of MPI:}
\begin{itemize}
  \item \textbf{Direct Communication}: MPI provides direct point-to-point communication without the overhead of RPC marshalling/unmarshalling
  \item \textbf{Simpler Protocol}: No need to define interface specifications (like XDR files in RPC)
  \item \textbf{Better Performance}: Lower latency for simple data transfers without RPC overhead
  \item \textbf{No Code Generation}: No need for code generation tools like \texttt{rpcgen}
\end{itemize}

\textbf{Disadvantages of MPI:}
\begin{itemize}
  \item \textbf{Less Abstraction}: RPC provides a more abstract interface (function calls) compared to MPI's message-passing model
  \item \textbf{No Interface Definition}: RPC's interface definition files provide better documentation and type safety
  \item \textbf{Platform Dependency}: MPI is primarily used in HPC environments, while RPC is more general-purpose
\end{itemize}

\section{Personal Work}

All tasks for this practical work were completed individually by Pham Huu Minh (23BI14302):

\begin{itemize}
  \item Research and selection of MPI implementation (OpenMPI)
  \item Installation and configuration of OpenMPI on Kali Linux
  \item Design of the MPI-based file transfer architecture
  \item Implementation of the sender process (Rank 0) logic
  \item Implementation of the receiver process (Rank 1) logic
  \item Implementation of file I/O helper functions
  \item Code compilation and debugging
  \item Testing with various file sizes and types
  \item Comparison analysis with TCP Socket and RPC implementations
  \item Documentation and report writing
\end{itemize}

\section{Conclusion}

This practical work successfully demonstrated the implementation of a file transfer system using the Message Passing Interface (MPI). The system was able to transfer files between two MPI processes using OpenMPI, with Rank 0 acting as the sender and Rank 1 as the receiver.

The implementation showed that MPI provides a clean and efficient way to implement distributed file transfer, with advantages in terms of code simplicity and performance compared to low-level socket programming. However, it requires the MPI runtime environment and has a different programming model compared to RPC.

The lab successfully demonstrated three different approaches to distributed file transfer:
\begin{itemize}
  \item \textbf{TCP Sockets}: Low-level, flexible, but requires manual network programming
  \item \textbf{RPC}: High-level abstraction with interface definitions, but with marshalling overhead
  \item \textbf{MPI}: Efficient message-passing model, ideal for HPC environments, but requires MPI runtime
\end{itemize}

Each approach has its own strengths and is suitable for different use cases and environments.

\end{document}
